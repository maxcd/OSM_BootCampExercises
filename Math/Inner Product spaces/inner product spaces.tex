\documentclass[letterpaper,12pt]{article}
\usepackage{array}
\usepackage{threeparttable}
\usepackage{geometry}
\geometry{letterpaper,tmargin=1in,bmargin=1in,lmargin=1.25in,rmargin=1.25in}
\usepackage{fancyhdr,lastpage}
\pagestyle{fancy}
\lhead{}
\chead{}
\rhead{}
\lfoot{\footnotesize\textsl{OSM Lab, Summer 2017, Math PS \#2}}
\cfoot{}
\rfoot{\footnotesize\textsl{Page \thepage\ of \pageref{LastPage}}}
\renewcommand\headrulewidth{0pt}
\renewcommand\footrulewidth{0pt}
\usepackage[format=hang,font=normalsize,labelfont=bf]{caption}
\usepackage{amsmath}
\usepackage{amssymb}
\usepackage{amsthm}
\usepackage{natbib}
\usepackage{setspace}
\usepackage{float,color}
\usepackage[pdftex]{graphicx}
\usepackage{hyperref}
\hypersetup{colorlinks,linkcolor=red,urlcolor=blue,citecolor=red}
\theoremstyle{definition}
\newtheorem{theorem}{Theorem}
\newtheorem{acknowledgement}[theorem]{Acknowledgement}
\newtheorem{algorithm}[theorem]{Algorithm}
\newtheorem{axiom}[theorem]{Axiom}
\newtheorem{case}[theorem]{Case}
\newtheorem{claim}[theorem]{Claim}
\newtheorem{conclusion}[theorem]{Conclusion}
\newtheorem{condition}[theorem]{Condition}
\newtheorem{conjecture}[theorem]{Conjecture}
\newtheorem{corollary}[theorem]{Corollary}
\newtheorem{criterion}[theorem]{Criterion}
\newtheorem{definition}[theorem]{Definition}
\newtheorem{derivation}{Derivation} % Number derivations on their own
\newtheorem{example}[theorem]{Example}
\newtheorem{exercise}[theorem]{Exercise}
\newtheorem{lemma}[theorem]{Lemma}
\newtheorem{notation}[theorem]{Notation}
\newtheorem{problem}[theorem]{Problem}
\newtheorem{proposition}{Proposition} % Number propositions on their own
\newtheorem{remark}[theorem]{Remark}
\newtheorem{solution}[theorem]{Solution}
\newtheorem{summary}[theorem]{Summary}
%\numberwithin{equation}{section}
\bibliographystyle{aer}
\newcommand\ve{\varepsilon}
\newcommand\boldline{\arrayrulewidth{1pt}\hline}

\begin{document}

\begin{flushleft}
   \textbf{\large{Math, Problem Set \#2, Inner Prouct Spaces}} \\[5pt] Zachery \\[5pt]
   Due Wednesday, July 5 at 8:00am
\end{flushleft}
\textbf{Homework:} 1, 2, 3, 8, 9, 10, 11, 16, 17, 23, 24, 26, 28, 29, 30, 37, 38, 39, 40, 44, 45, 46,
47, 48, 50 at the end of Chapter 3 of Humpherys et al. (2017)
\begin{enumerate}
\item[3.1)] 
\begin{itemize}
\item[i)]
\begin{align*}
&\frac{1}{4} \left( ||x+y||^2 - ||x-y||^2 \right) \\
=&\frac{1}{4} \left(||x||^2  +||y||^2 + 2||x||\,||y||\,cos(\theta) - [||x||^2  +||y||^2 - 2||x||\,||y|| \, cos(\theta) ] \right) \\
=& \frac{1}{4} \left( 4 ||x||\, ||y|| \, cos(\theta) \right) \quad by \,  definition \, of \, cos(\theta)\\
=&||x||\, ||y|| \, \frac{\langle x,y\rangle}{||x||\, ||y|| \,} \\
=& \langle x ,y\rangle
\end{align*}
\item[ii)] 
\begin{align*}
&\frac{1}{2} \left( ||x+y||^2 + ||x-y||^2 \right) \\
=& \frac{1}{2} \left( ||x||^2 \, ||y||^2 + 2||x||\,||y||\,cos(\theta)  +    ||x||^2  +||y||^2 - 2||x||\,||y|| \, cos(\theta) \right) \\
=& \frac{1}{2} 2 \left( ||x||^2+||y||^2  \right) \\
=& ||x||^2+||y||^2 
\end{align*}
\end{itemize}
\item[3.2)]
\begin{align*}
&\frac{1}{4} \left( ||x+y||^2 - ||x-y||^2  + i||x-iy||^2 - i||x+iy||^2 \right) \\
&\frac{1}{4} \left( ||x+y||^2 - ||x-y||^2  - i(||x+iy||^2 - ||x-iy||^2  )\right) \\
=& \frac{1}{4} ( ||x||^2 +\langle x, y \rangle +\langle  y ,x
\rangle +||y||^2 \\ 
\quad & - ||x||^2 +\langle x, y \rangle +\langle  y ,x\rangle -||y||^2 \\
\quad &-i( ||x||^2 +\langle x, iy \rangle +\langle  iy ,x
\rangle +||y||^2 \\ 
\quad & - ||x||^2 +\langle x, iy \rangle +\langle  iy ,x\rangle -||y||^2 ))
\\
=& \frac{1}{4}( 2 \langle x, y \rangle + 2\langle y, x \rangle \\
\quad &-i(2i\langle x, y \rangle  - 2i\langle y,x \rangle ) \\
=& \frac{1}{4}(2 \langle x, y \rangle + 2\langle y, x \rangle \\
\quad & + 2\langle x, y \rangle  - 2\langle y,x \rangle) \\
=& \frac{1}{4}(4\langle x, y \rangle) = \langle x, y \rangle
\end{align*}
\item[3.3)] \begin{eqnarray*}\theta = cos^{-1} \left( \frac{\langle f, g \rangle}{||f|| \, ||g||} \right) \end{eqnarray*}
for i) and ii) we have \begin{eqnarray*}
\langle f, g \rangle = \int_o^1 x^6 \, dx = \frac{1}{7}\end{eqnarray*}
and for i) we further have:
\begin{align*}
||g|| =&  ||x||= \sqrt{\langle g, g \rangle} = \left( \frac{1}{3} \right)^\frac{1}{2}\\
||f|| =&= ||x^5||  \sqrt{\langle f, f \rangle} =\left( \frac{1}{11} \right)^\frac{1}{2}\\
\theta =& \cos^{-1}\left( \frac{\sqrt{33}}{7} \right)
\end{align*}
and for ii)
\begin{align*}
||g|| =&  ||x^2||= \sqrt{\langle g, g \rangle} = \left( \frac{1}{5} \right)^\frac{1}{2}\\
||f|| =& =||x^4||= \sqrt{\langle f, f \rangle} =\left( \frac{1}{9} \right)^\frac{1}{2}\\
\theta =& \cos^{-1}\left( \frac{\sqrt{45}}{7} \right)
\end{align*}
\item[3.8)]
\begin{itemize}
\item[i)]\begin{align*}
\langle cos^2(t) \rangle = \frac{1}{\pi} \int_{-\pi}^{\pi} cos^2(t)dt =& \frac{1}{\pi} \int_{-\pi}^{\pi} \frac{1+cos(2t)}{2} dt = \frac{1}{2\pi} \left( \int_{-\pi}^{\pi}1dt + \int_{-\pi}^{\pi} cos(t)dt \right)\\
=& \frac{1}{2\pi} \left[ t \right]_{-\pi}^{\pi} = \frac{1}{2\pi} [\pi + \pi] = 1 \\
\langle cos^2(2t) \rangle = \frac{1}{\pi} \int_{-\pi}^{\pi} cos^2(2t)dt =& \frac{1}{\pi} \int_{-\pi}^{\pi} \frac{1+cos(4t)}{2} dt  \quad and \, because \, cos(4t)=cos(2t)\\
=&\frac{1}{2\pi} \left( \int_{-\pi}^{\pi}1dt + \int_{-\pi}^{\pi} cos(t)dt \right)\\
=& \frac{1}{2\pi} \left[ t \right]_{-\pi}^{\pi} = \frac{1}{2\pi} [\pi + \pi] = 1 \\
\langle sin^2(t) \rangle = \frac{1}{\pi} \int_{-\pi}^{\pi} sin^2(t)dt =& \frac{1}{\pi} \int_{-\pi}^{\pi} \frac{1-cos(2t)}{2} dt = \frac{1}{2\pi} \left( \int_{-\pi}^{\pi}1\, dt - \int_{-\pi}^{\pi} cos(t)dt \right)\\
=& \frac{1}{2\pi} \left[ t \right]_{-\pi}^{\pi} = \frac{1}{2\pi} [\pi + \pi] = 1 \\
\langle sin^2(2t) \rangle = \frac{1}{\pi} \int_{-\pi}^{\pi} sin^2(2t)dt =& \frac{1}{\pi} \int_{-\pi}^{\pi} \frac{1-cos(4t)}{2} dt  \quad and \, because \, sin(4t)=cos(2t)\\
=&\frac{1}{2\pi} \left( \int_{-\pi}^{\pi}1dt + \int_{-\pi}^{\pi} cos(t)dt \right)\\
=& \frac{1}{2\pi} \left[ t \right]_{-\pi}^{\pi} = \frac{1}{2\pi} [\pi + \pi] = 1 \\
\end{align*}
\begin{align*}
\langle cos(t,)sin(t) \rangle  =& \int_{-\pi}^{\pi}cos(t)sin(t) dt = 0 \\
\langle cos(t), cos(2t)\rangle  =& \int_{-\pi}^{\pi}cos(t) sin(2t) dt = 0 \\
\langle cos(t), sin(2t) \rangle =& \int_{-\pi}^{\pi} cos(t) sin(2t) dt = 0 \\
\langle sin(t) ,cos(2t)\rangle =& \int_{-\pi}^{\pi}sin(t) cos(2t) dt = 0 \\
\langle sin(t) ,sin(2t)\rangle =& \int_{-\pi}^{\pi}sin(t) cos(2t) dt = 0 \\
\langle cos(2t),sin(2t) \rangle  =& \int_{-\pi}^{\pi}cos(2t) sin(2t) dt = 0 \\
\end{align*}
\item[ii)]
\begin{align*}
||t|| = \langle t,t \rangle^{0.5}= \left( \frac{1}{\pi}\int_{-\pi}^{\pi} t^2 \, dt \right)^{0.5}= \left(\frac{1}{\pi} \times \frac{2 \pi^3}{3} \right)^{0.5} = \sqrt{\frac{2}{3}} \pi
\end{align*}
\item[iii)]
\begin{align*}
proj_X(cos(3t)) =& \sum_{i=1}^4 \langle x_i, cos(3t)\rangle x_i \\
 =& \langle cos(t), cos(3t)\rangle cos(t) + \langle sin(t), cos(3t)\rangle sin(t) \\
\quad & + \langle cos(2t), cos(3t)\rangle cos(2t) + \langle sin(2t), cos(3t)\rangle sin(2t) \\
=& 0 + 0 +0 +0 = 0
\end{align*}
\item[iv)]
\begin{align*}
proj_X(t)=& \sum_{i=1}^4 \langle x_i, t \rangle x_i\\
 =& \frac{1}{\pi}\int_{-\pi}^{\pi} t\, cos(t)\, dt\, cos(t) +\frac{1}{\pi} \int_{-\pi}^{\pi} t\, sin(t)\, dt \,sin(t) \\
 \quad &+ \frac{1}{\pi}\int_{-\pi}^{\pi} t \,cos(2t)\, dt\, cos(2t)  + \frac{1}{\pi}\int_{-\pi}^{\pi} t\, sin(2t)\, dt\, sin(2t) \\
 =& \, 0 + \frac{2\pi}{\pi} sin(t) + 0 + -\frac{\pi}{\pi} sin(2t)\\
 =&\, 2sin(t)-sin(2t)
\end{align*}
\end{itemize}
\end{enumerate}


\end{document}

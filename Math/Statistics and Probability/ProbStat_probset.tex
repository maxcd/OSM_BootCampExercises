\documentclass[letterpaper,12pt]{article}
\usepackage{array}
\usepackage{threeparttable}
\usepackage{geometry}
\geometry{letterpaper,tmargin=1in,bmargin=1in,lmargin=1.25in,rmargin=1.25in}
\usepackage{fancyhdr,lastpage}
\pagestyle{fancy}
\lhead{}
\chead{}
\rhead{}
\lfoot{\footnotesize\textsl{OSM Lab, Summer 2017, Math PS \#1}}
\cfoot{}
\rfoot{\footnotesize\textsl{Page \thepage\ of \pageref{LastPage}}}
\renewcommand\headrulewidth{0pt}
\renewcommand\footrulewidth{0pt}
\usepackage[format=hang,font=normalsize,labelfont=bf]{caption}
\usepackage{amsmath}
\usepackage{amssymb}
\usepackage{amsthm}
\usepackage{natbib}
\usepackage{setspace}
\usepackage{float,color}
\usepackage[pdftex]{graphicx}
\usepackage{hyperref}
\hypersetup{colorlinks,linkcolor=red,urlcolor=blue,citecolor=red}
\theoremstyle{definition}
\newtheorem{theorem}{Theorem}
\newtheorem{acknowledgement}[theorem]{Acknowledgement}
\newtheorem{algorithm}[theorem]{Algorithm}
\newtheorem{axiom}[theorem]{Axiom}
\newtheorem{case}[theorem]{Case}
\newtheorem{claim}[theorem]{Claim}
\newtheorem{conclusion}[theorem]{Conclusion}
\newtheorem{condition}[theorem]{Condition}
\newtheorem{conjecture}[theorem]{Conjecture}
\newtheorem{corollary}[theorem]{Corollary}
\newtheorem{criterion}[theorem]{Criterion}
\newtheorem{definition}[theorem]{Definition}
\newtheorem{derivation}{Derivation} % Number derivations on their own
\newtheorem{example}[theorem]{Example}
\newtheorem{exercise}[theorem]{Exercise}
\newtheorem{lemma}[theorem]{Lemma}
\newtheorem{notation}[theorem]{Notation}
\newtheorem{problem}[theorem]{Problem}
\newtheorem{proposition}{Proposition} % Number propositions on their own
\newtheorem{remark}[theorem]{Remark}
\newtheorem{solution}[theorem]{Solution}
\newtheorem{summary}[theorem]{Summary}
%\numberwithin{equation}{section}
\bibliographystyle{aer}
\newcommand\ve{\varepsilon}
\newcommand\boldline{\arrayrulewidth{1pt}\hline}

\begin{document}

\begin{flushleft}
   \textbf{\large{Math, Problem Set \#1, Probability Theory}} \\[5pt]
   OSM Lab, Karl Schmedders \\[5pt]
   Due Monday, June 26 at 8:00am
\end{flushleft}

\vspace{5mm}

\begin{enumerate}
	\item {\bf Exercises from chapter.} Do the following exercises in Chapter 3 of \citet{HJ17}: 3.6, 3.8, 3.11, 3.12 (watch this movie \href{https://www.youtube.com/watch?v=Zr_xWfThjJ0}{clip}), 3.16, 3.33, 3.36. \\
	
	\textbf{Solutions:}\\
	3.6\\
 Assumptions: 
\begin{enumerate}
\item  $\Omega = \cup_{i\in I} B_i$ 
\item $B_i\cap B_j = 0$
\end{enumerate}  
Proof: 
\begin{align*}
P(A) =  P(\Omega \cap A) = P(\cup_{i\in I} B_i \cap A)
 = P(\cup_{i \in I} (B_i \cap A)) = \sum_{i \in I} P(A \cap B_i)
\end{align*} \\
3.8 \\ 
First, if the events $E_k$ are independent, than there complements $E_k^c$ are independet, too.
\begin{align*}
1-\prod_{k=1}^{n}(1 - P(E_k)) =& 1 - \prod_{k=1}^{n}\left( P(E) \right)^c \\
=& 1 -P(\cap^n_{k=1} E^c_k) 
\end{align*}
and by De Morgans law we get
\begin{align*}
=& 1 - P((\cup^n_{k=1} E_k)^c) \\
=& P(\cap^n_{k=1} E_k)) 
\end{align*} \\
3.11 TODO\\
3.12 \\
In the first round the unconditional  probability of picking the door with the car out of the three door is is just $P(car) = \frac{1}{3}$. After having observed one goat the conditional probability of choosing the door with the car is $P(car|seen\, one\, goat) = \frac{1}{2}$ because one chooses randomly out of two possible choices.\\
When there are 10 doors $P(car)=\frac{1}{10}$ and $P(car|seen\, 8\, goats) = \frac{1}{2}$ because, again, you are randomly choosing one out of two possible choices conditional on the 8 goats you have already observed.\\
3.16\\
\begin{align*}
Var[X] =& E\left[ (X-\mu)^2 \right] = E\left[ X^2 - 2 X \mu + \mu^2 \right] \\
=& E[X^2] - 2E[X]\mu +\mu^2 =  E[X^2] - 2\mu^2 +\mu^2 \\
=&  E[X^2] - \mu^2 
\end{align*}\\
3.33\\

Since $B \sim Binom(n, p)$, $E[B]=np$ and $Var[B]= \sigma^2 = p(1-p)$
\begin{align*}
P\left(|\frac{B}{n}-p| \geq \varepsilon \right) =& P \left( |B-np|  \geq n \varepsilon \right) \\
\end{align*}
And by the Chebyshev's Inequality we get
\begin{align*}
P \left( |B-np|  \geq n \varepsilon \right) \leq& \frac{np(1-p)}{n^2 \varepsilon^2} \\
\leq& \frac{p(1-p}{n \varepsilon^2}
\end{align*}\\

3.36\\
Since $X_i \sim Bernoulli(p)\, for \,  i \in (1, 6242)$ with $E[x]=p=0.801$ and $Var[X] = \sigma^2 =  p(1-p) = 0.199 \times 0.801$, let the number of students actually enrolling be $S = \sum^6242_i=1 X_i$, then the variable $\frac{S-6242p}{\sigma\sqrt{6242}} \sim N(0,1)$. Then the probability that more than 5500 students will enroll is given by $1-P(x < \frac{5500-6242p}{\sigma\sqrt{6242}}) = 0$. This probability is practically zero.
	\item Construct examples of events $A$, $B$, and $C$, each of probability strictly between 0 and 1, such that
   		\begin{itemize}
			\item[(a)] $P(A  \cap B) = P(A)P(B)$, $P(A  \cap C) = P(A)P(C)$, $P(B  \cap C) = P(B)P(C)$, but $P(A  \cap B \cap C) \neq P(A)P(B)P(C)$.
			\item[(b)] $P(A  \cap B) = P(A)P(B)$, $P(A  \cap C) = P(A)P(C)$, $P(A  \cap B \cap C) = P(A)P(B)P(C)$, but $P(B  \cap C) \neq P(B)P(C)$. (Hint: You can let $\Omega$ be a set of eight equally likely points.)
		\end{itemize}
		
\textbf{Solutions:}\\
\begin{itemize}
\item[(a)]  let $\Omega={1,2,3,4,5,6,7,8}$ with $P(x=x_i)=\frac{1}{8}\,  \forall \, x_i \in \Omega$ and with the events
\begin{align*}
A &= \lbrace 1, 2, 3, 4\rbrace \\
 B&=\lbrace1, 2, 5, 6 \rbrace \\
 C&=\lbrace 3, 4, 5, 6 \rbrace
\end{align*}
Such that $P(A)=P(B)=P(C)=\frac{4}{8}$ and the intersections $A  \cap B$, $A  \cap B$, $B \cap C $  each have two elements leading to  $P(A  \cap B) = P(A  \cap C) = =P(B \cap C )= P(A)P(C) = P(A)P(B) = P(B)P(C) = \frac{2}{8}$. Furthermore, $P(A  \cap B \cap C) = P(\emptyset) = 0 \neq P(A)P(B)P(C) = \frac{1}{8}$.
$\emptyset$
\item[(b)] for the same $\Omega$ as in (a) be 
\begin{align*}
A &= \lbrace 1, 2, 3, 4\rbrace \\
 B&=\lbrace3, 4, 6, 7 \rbrace \\
 C&=\lbrace 1, 2, 4, 5 \rbrace
\end{align*}
Such that $P(A)=P(B)=P(C)=\frac{4}{8}$ and the intersections $A  \cap B$ and $A  \cap B$  each have two elements leading to  $P(A  \cap B) = P(A  \cap C) = P(A)P(C) = P(A)P(B) = \frac{2}{8}$. Furthermore, $P(B  \cap C) = P(x=4) = \frac{1}{8}\neq P(B)P(C) = \frac{2}{8}$ and $P(A  \cap B \cap C) = P(4) = \frac{1}{8} = P(A)P(B)P(C)$.
\end{itemize}

   	\item Prove that Benford's Law is, in fact, a well-defined discrete probability distribution.\\

\textbf{Solutions:} 
Benford's law holds for the sample space
\begin{itemize}
\item[1.] since $2 \geq (1+\frac{1}{d}) \geq (1+\frac{1}{9}) \, \forall \, d \in \Omega$ it follows that $0 \geq log_{10}(1+\frac{1}{d}) \geq 1$ holds.
\item[2.] \begin{align*}
&\sum_{d=1}^9 log_{10} \left(1+\frac{1}{d}\right )=log_{10}\left(\prod^9_{d=1}1+\frac{1}{d} \right)\\
&=log_{10}(
  2 \times
1.5\times
1.33\times
1.25\times
1.2\times
1.167\times
1.143\times
1.125\times
1.111\times)\\
 &= log_{10}(10)=1
\end{align*}
\item[3.] for any events $d_i$ and $d_j$ with $i\neq j$, the intersection $d_i \cap d_j = \emptyset$ (i.e. they are pairwise disjoint) such that $P(d_i \cup d_j)=P(d_i)+P(d_j)$
\end{itemize}

   	\item A person tosses a fair coin until a tail appears for the first time. If the tail appears on the $n$th flip, the person wins $2^n$ dollars. Let the random variable $X$ denote the player's winnings.
		\begin{itemize}
			\item[(a)] (St. Petersburg paradox) Show that $E[X]= + \infty$.
			\item[(b)] Suppose the agent has log utility. Calculate $E[\ln X]$.\\
			

Let $X$ be the payoff from the St. Petersburg game, than its expected value can be calculated by
\begin{align*}
E[X] = \sum_{k=1}^\infty \left(\frac{1}{2} \right)^k 2^{k} =1+ 1+ 1+ 1+ \,... = +\infty
\end{align*}
When the payoff of one game is weighted with the utility function $u(x)=ln(x)$, the expected utility is given by
\begin{align*}
E[u(X)] =& \sum_{k=1}^\infty \left(\frac{1}{2}\right)^k  ln(2^{k}) =  ln(2) \sum_{k=1}^\infty \frac{k}{2^k} = ln(2) S \\ 
\end{align*} where
\begin{align*}
S=& \sum_{k=1}^\infty \frac{k}{2^k} = S -\frac{1}{2}S+ \frac{1}{2}S = S-\sum_{k=1}^\infty \frac{1}{2}\frac{k}{2^k}+ \frac{1}{2}S = S-\sum_{k=1}^\infty \frac{k}{2^{k+1}}+ \frac{1}{2}S \\
=&  \sum_{k=1}^\infty \frac{k}{2^k}- \sum_{k=2}^\infty \frac{k-1}{2^{k}}+ \frac{1}{2}S = \sum^{\infty}_{k=1} \frac{1}{2^k} + \frac{1}{2}S = 1 + \frac{1}{2}S \Rightarrow S=2\\
\end{align*}
so,
\begin{align*}
E[X] = 2ln(2)
\end{align*}
		\end{itemize}
	\item (Siegel's paradox) Suppose the exchange rate between USD and CHF is 1:1. Both a U.S. investor and a Swiss investor believe that a year from now the exchange rate will be either $1.25:1$ or $1:1.25$, with each scenario having a probability of 0.5. Both investors want to maximize their wealth in their respective home currency (a year from now) by investing in a risk-free asset; the risk-free interest rates in the U.S. and in Switzerland are the same. Where should the two investors invest?

\item Consider a probability measure space with $\Omega = [0,1]$.
		\begin{itemize}
			\item[(a)] Construct a random variable $X$ such that $E[X] < \infty$ but $E[X^2] = \infty$.
			\item[(b)] Construct random variables $X$ and $Y$ such that $P(X>Y)>\frac{1}{2}$ but $E[X]<E[Y]$.
			\item[(c)] Construct random variables $X$, $Y$, and $Z$ such that\\ $P(X>Y) P(Y>Z) P(X>Z) > 0$ and 						$E(X)=E(Y)=E(Z)=0$.
		\end{itemize}

	\item Let the random variables $X$ and $Z$ be independent with $X \sim N(0,1)$ and $P(Z=1)=P(Z=-1)=\frac{1}{2}$. 			Define $Y= XZ$ as the product of $X$ and $Z$. Prove or disprove each of the following statements.
		\begin{itemize}
			\item[(a)] $Y \sim N(0,1)$.
			\item[(b)] $P(|X|=|Y|)=1$.
			\item[(c)] $X$ and $Y$ are not independent.
			\item[(d)] $Cov[X,Y]=0$.
			\item[(e)] If $X$ and $Y$ are normally distributed random variables with $Cov[X,Y]=0$, then $X$ and $Y$ 					must be dependent.
		\end{itemize}

	\item Let the random variables $X_i$, $i=1,2,\ldots,n,$ be i.i.d.\ having the uniform distribution on $[0,1]$, denoted $X_i \sim U[0,1]$. Consider the random variables $m=\min\{X_1,X_2,\ldots,X_n\}$ and $M=\max\{X_1,X_2,\ldots,X_n\}$. For both random variables $m$ and $M$, derive their respective cumulative distribution (cdf), probability density function (pdf), and expected value.

	\item You want to simulate a dynamic economy (e.g., an OLG model) with two possible states in each period, a ``good'' state and a ``bad'' state. In each period, the probability of both shocks is $\frac{1}{2}$. Across periods the shocks are independent. Answer the following questions using the Central Limit Theorem and the Chebyshev Inequality.
		\begin{itemize}
			\item[(a)] What is the probability that the number of good states over 1000 periods differs from 500 by at most 2\%?
			\item[(b)] Over how many periods do you need to simulate the economy to have a probability of at least 0.99 that the proportion of good states differs from $\frac{1}{2}$ by less than 1\%?
		\end{itemize}

	\item If $E[X]<0$ and $\theta \neq 0$ is such that $E[e^{\theta X}]=1$, prove that $\theta > 0$.
\end{enumerate}

\vspace{25mm}

\bibliography{ProbStat_probset}

\end{document}
